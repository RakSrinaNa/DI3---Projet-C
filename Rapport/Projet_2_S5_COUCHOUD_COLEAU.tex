\documentclass{EPUProjetPeiP}
\usepackage{tcolorbox}
\newtcbox{\hl}[1][yellow]{on line, arc=7pt,colback=#1!10!white,colframe=#1!50!black,
  before upper={\rule[-3pt]{0pt}{10pt}},boxrule=1pt, boxsep=0pt,left=6pt,
  right=6pt,top=2pt,bottom=2pt}
\usepackage{listings}
\usepackage{float}
\usepackage{color}
\usepackage{graphicx}
\usepackage{wrapfig}
\usepackage[justification=centering]{caption}
\definecolor{dkgreen}{rgb}{0,0.6,0}
\definecolor{gray}{rgb}{0.5,0.5,0.5}
\definecolor{mauve}{rgb}{0.58,0,0.82}
\lstset{
  language=XML,
  morekeywords={element,category,encoding,
    xs:schema,xs:element,xs:complexType,xs:sequence,xs:attribute}
}
\lstset{frame=tb,
  language=Java,
  aboveskip=3mm,
  belowskip=3mm,
  showstringspaces=false,
  columns=flexible,
  basicstyle={\small\ttfamily},
  numbers=none,
  numberstyle=\tiny\color{gray},
  keywordstyle=\color{blue},
  commentstyle=\color{dkgreen},
  stringstyle=\color{mauve},
  breaklines=true,
  breakatwhitespace=true,
  tabsize=3
}
\lstset{literate=
  {á}{{\'a}}1 {é}{{\'e}}1 {í}{{\'i}}1 {ó}{{\'o}}1 {ú}{{\'u}}1
  {Á}{{\'A}}1 {É}{{\'E}}1 {Í}{{\'I}}1 {Ó}{{\'O}}1 {Ú}{{\'U}}1
  {à}{{\`a}}1 {è}{{\`e}}1 {ì}{{\`i}}1 {ò}{{\`o}}1 {ù}{{\`u}}1
  {À}{{\`A}}1 {È}{{\'E}}1 {Ì}{{\`I}}1 {Ò}{{\`O}}1 {Ù}{{\`U}}1
  {ä}{{\"a}}1 {ë}{{\"e}}1 {ï}{{\"i}}1 {ö}{{\"o}}1 {ü}{{\"u}}1
  {Ä}{{\"A}}1 {Ë}{{\"E}}1 {Ï}{{\"I}}1 {Ö}{{\"O}}1 {Ü}{{\"U}}1
  {â}{{\^a}}1 {ê}{{\^e}}1 {î}{{\^i}}1 {ô}{{\^o}}1 {û}{{\^u}}1
  {Â}{{\^A}}1 {Ê}{{\^E}}1 {Î}{{\^I}}1 {Ô}{{\^O}}1 {Û}{{\^U}}1
  {œ}{{\oe}}1 {Œ}{{\OE}}1 {æ}{{\ae}}1 {Æ}{{\AE}}1 {ß}{{\ss}}1
  {ű}{{\H{u}}}1 {Ű}{{\H{U}}}1 {ő}{{\H{o}}}1 {Ő}{{\H{O}}}1
  {ç}{{\c c}}1 {Ç}{{\c C}}1 {ø}{{\o}}1 {å}{{\r a}}1 {Å}{{\r A}}1
  {€}{{\EUR}}1 {£}{{\pounds}}1
}
\newcommand{\pic}[3]{
	\begin{figure}[H]
		\begin{center}
			\includegraphics[scale=#2]{doc/#1}
		\end{center}
		\caption{#3}
	\end{figure}
}
\newcommand{\piclab}[4]{
	\begin{figure}[H]
		\begin{center}
			\includegraphics[scale=#2]{doc/#1}
		\end{center}
		\caption{\label{#4}#3}
	\end{figure}
}
\newcommand{\picsidelab}[6]{
	\begin{wrapfigure}{#5}{#6}
		\includegraphics[scale=#2]{doc/#1}
		\caption{\label{#4}#3}
	\end{wrapfigure}
}
\makeindex

\title[Projet sac à dos]{Projet tutoré 2: Sac à dos}

\projet{S4}

\author{Thomas Couchoud\\
\noindent[\url{thomas.couchoud@etu.univ-tours.fr}]\\
Victor Coleau\\
\noindent[\url{victor.coleau@etu.univ-tours.fr}]}

\encadrant{Yannick Kergosien\\ %
\noindent[\url{yannick.kergosien@univ-tours.fr}]~\\
Polytech Tours\\
Département DI\\~ %
}

%%%%%%%%%%%%%%%%%%%%%%%%%%%%%%%%%%%%%%%%%%%%%%%%%%%%%%%%%%%%%%%%%%%%%%%%%%%%%%%%%%%%%%%%%%
\begin{document}

\maketitle

\pagenumbering{roman}
\setcounter{page}{0}
{
\setlength{\parskip}{0em}

\tableofcontents

%\listoffigures
%rq1 : si vous n'avez pratiquement pas de figures, laissez la ligne précédente en commentaire

%\listoftables
%rq1 : si vous n'avez pratiquement pas de tables, laissez la ligne précédente en commentaire
}


\start
%%%%%%%%%%%%%%%%%%%%%%%%%%%%%%%%%%%%%%%%%%%%%%%%%%%%%%%%%%%%%%%%%%%%%%%%%%%%%%%%%%%%%%%%%%

\chapter*{Introduction}
Le projet que nous avons choisi est celui portant sur le problème du sac à dos (\textit{Knapsack problem}). Celui-ci se généralise très simplement et donne lieu à de nombreux problèmes analogues.

Dans notre cas, nous devons remplir un sac à dos d'objets. Chaque objet a une certaine valeur prédéfinie ainsi que des "poids" dans différentes dimensions. On peut imaginer le cas où l'on tenterait de remplir sa valise pour partir en voyage. Chaque objet a une valeur selon l'importance qu'on lui donne ainsi que des "poids" qui pourraient être la place qu'il occupe, son poids réel etc.. L'idée ici est d'essayer de maximiser la valeur que nous emportons avec nous sachant que notre valise est limitée en poids et taille.

D'un point de vu mathématique, on peut modéliser ceci simplement:
\begin{itemize}
	\item $X$ Un vecteur définissant quels items sont dans le sac ou pas (ex: $(0; 0; 1)$ définira un sac avec seulement le 3ème item de pris).
	\item $W_j$ Le poids maximum que le sac peut supporter dans la dimension $i$.
	\item $w_{i,j}$ Le poids du $i$ème item dans la $j$ème dimension.
	\item $v_i$ La valeur du $i$ème item.\\
\end{itemize}
Les contraintes sont: $\forall j\in [0...m], \sum_{i=0}^nx_iw_{i,j}\leq W_j$\\
On appellera la fonction objectif $z(X)$ la fonction donnant la valeur d'un sac: $z(X)=\sum_{i=0}^mx_iv_i$

\chapter{Structure du projet}
Le C ayant un langage rassemblant tous les fichiers en un lors de la compilation, il est nécessaire de choisir judicieusement ses noms de fonctions afin d'éviter les duplicatas. Dans notre cas nous avons choisi un formatage simple: [Nom du .c / Nom de la structure]\_[Nom de la fonction]. Nous aurons donc des fonctions du type \textit{population\_create(...)} ou bien \textit{metaheuristicGenetic\_search(...)}.

\section{Fichiers}
Concernant l'organisation des fichiers en eux même, chaque type d'entre eux est localisé à un endroit différent. En effet nous avions commencé par mettre tous nos .c et .h dans un même dossier. Cependant, le projet grandissant assez vite, il a rapidement arrivé un stade où l'on se perd. Pour cela nous avons décidé de séparer les .h des .c puisque nous travaillons principalement sur les .c. Cela permet de s'y retrouver plus aisément. Ainsi la structure de notre dossier source est la suivante:
\begin{itemize}
	\item src $\longrightarrow$ Le dossier racine contenant nos .c pour le programme
	\begin{itemize}
		\item headers $\longrightarrow$ Le dossier contenant nos headers pour le programme
		\item unit $\longrightarrow$ Le dossier contenant nos .c pour les tests unitaires
		\begin{itemize}
			\item headers $\longrightarrow$ Le dossier contenant nos headers pour les tests unitaires\\
		\end{itemize}	
	\end{itemize}
\end{itemize}

Intéressons-nous au dossier src. Nous avons décider de créer des fichiers spécifiques pour certaines structures ainsi que leurs fonctions. Celles-ci sont:
\begin{itemize}
	\item Bag $\longrightarrow$ Représentant le contenu du sac pour une solution indirecte.
	\item SolutionDirect $\longrightarrow$ Représentant une solution directe.
	\item SolutionIndirect $\longrightarrow$ Représentant une solution indirecte.
	\item Solution $\longrightarrow$ Représentant l'union d'une solution directe et indirecte.
	\item Instance $\longrightarrow$ Représentant une instance.
	\item Item $\longrightarrow$ Représentant un élément du sac.\\
\end{itemize}

A coté de cela, nous avons plusieurs fichiers n'étant que de simples regroupements de fonctions selon leur utilisation:
\begin{itemize}
	\item Parser $\longrightarrow$ Regroupant les différentes fonctions afin de lire un fichier.
	\item Heuristic $\longrightarrow$ Regroupant les fonctions liées à la résolution grâce à une heuristique.
	\item Scheduler $\longrightarrow$ Regroupant les fonctions liées aux différents algorithmes pour l'heuristique.
	\item MetaheuristiqueLocal $\longrightarrow$ Regroupant les fonctions liées aux différents algorithmes pour la metaheuristique locale.
	\item MetaheuristiqueTabou $\longrightarrow$ Regroupant les fonctions liées aux différents algorithmes pour la metaheuristique tabou et contenant la structure Tabou.
	\item MetaheuristiqueGenetic $\longrightarrow$ Regroupant les fonctions liées aux différents algorithmes pour la metaheuristique génétique et contenant la structure Population.
\end{itemize}

\section{Structures}
\subsection{Parser}
Cette structure permet la lecture pas à pas d'un fichier. Celle-ci contient le chemin du fichier concerné, l'offset actuel de lecture, le nombre d'instances à lire et le nombre d'instances déjà lues.

Les fonctions suivantes lui sont associées:
\begin{itemize}
	\item parser\_create $\longrightarrow$ Permet de créer cette structure à partir du chemin d'un fichier.
	\item parser\_destroy $\longrightarrow$ Pour détruire la structure.
	\item parser\_getNextInstance $\longrightarrow$ Renvoi la prochaine instance du fichier ou NULL si l'on a atteint la fin.
\end{itemize}

\chapter{Parser}
L'une des premières parties que nous devions réaliser est le parser. Lors de cette dernière, un choix important a du se faire: lisons-nous toutes les instances d'un fichier d'un seul coup ou lisons les nous une par une?

Nous avions initialement décidé de les lire toutes à la suite. En effet, ce choix était celui de la simplicité. Nous avons voulu commencer simple afin de pouvoir avancer sans attendre sur les autres tâches à faire. Puis rapidement nous avons implémenté la seconde méthode. Celle-ci nous paru plus adéquate pour nos utilisations car elle permet d'éviter une utilisation importante de la mémoire pour pas grand chose. Certes nous avons du créer une structure Parser qui sert principalement à conserver les informations de la dernière lecture, mais ce choix nous paru être le meilleur.

Ainsi, vous trouverez dans notre code les deux méthodes de lecture:
\begin{itemize}
	\item parser\_readAllFile $\longrightarrow$ qui renvoi toutes les instances d'un fichier
	\item parser\_create $\longrightarrow$ Permet l'utilisation de parser\_getNextInstance qui renvoi uniquement la prochaine Instance.
\end{itemize}

\chapter{Heuristique}
Dans le cadre des heuristiques, nous avons du implémenter nos propres critères de sélection. Nous allons ici vous en présenter deux.

Le premier se base sur l'algorithme de la dimension critique mais prend cette fois-ci en compte toutes les dimensions. Pour cela nous calculons pour l'item à l'index $i$ un ratio qui est $r_i=\sum_{j=0}^m\frac{w_j}{W_j}$. Ce ratio sert par la suite à calculer un score temporaire afin d'appliquer l'heuristique $score_i=\frac{v_i}{r_i}$. De cette manière, plus l'item remplira le sac, plus le diviseur sera important et par conséquent, l'item aura un score faible.

Le s

\chapter{Metaheuristiques}
\section{Local}
\section{Tabou}
\section{Genetique}

\chapter{Autres}

%--------------------------------------------------------------------------------
\chapter*{Conclusion}
\addcontentsline{toc}{chapter}{\numberline{}Conclusion}
\markboth{Conclusion}{}

\label{sec:conclusion}




%--------------------------------------------------------------------------------
%si on donne des annexes :
\appendix
\addcontentsline{toc}{part}{\numberline{}Annexes}

%--------------------------------------------------------------------------------
%index : attention, le fichier dindex .ind doit avoir le même nom que le fichier .tex
%\printindex

%--------------------------------------------------------------------------------
%page du dos de couverture :

\resume{Projet ayant pour objectif la réalisation d'un algorithme cherchant des solutions au problème du sac à dos multidimentionel.}

\motcles{sac à dos, algorithme, C, heuristique, metaheuristique, parser, directe, indirecte}

\abstract{Project which objective is to find solutions for the multidimentional Knapsack problem.}

\keywords{backpack, Knapsack, algorithm, C, heurictic, metaheuristic, parser, direct, indirect}

\makedernierepage

\end{document}

\documentclass{EPUProjetPeiP}
\usepackage{tcolorbox}
\newtcbox{\hl}[1][yellow]{on line, arc=7pt,colback=#1!10!white,colframe=#1!50!black,
  before upper={\rule[-3pt]{0pt}{10pt}},boxrule=1pt, boxsep=0pt,left=6pt,
  right=6pt,top=2pt,bottom=2pt}
\usepackage{listings}
\usepackage{float}
\usepackage{color}
\usepackage{graphicx}
\usepackage{wrapfig}
\usepackage[justification=centering]{caption}
\definecolor{dkgreen}{rgb}{0,0.6,0}
\definecolor{gray}{rgb}{0.5,0.5,0.5}
\definecolor{mauve}{rgb}{0.58,0,0.82}
\lstset{
  language=XML,
  morekeywords={element,category,encoding,
    xs:schema,xs:element,xs:complexType,xs:sequence,xs:attribute}
}
\lstset{frame=tb,
  language=Java,
  aboveskip=3mm,
  belowskip=3mm,
  showstringspaces=false,
  columns=flexible,
  basicstyle={\small\ttfamily},
  numbers=none,
  numberstyle=\tiny\color{gray},
  keywordstyle=\color{blue},
  commentstyle=\color{dkgreen},
  stringstyle=\color{mauve},
  breaklines=true,
  breakatwhitespace=true,
  tabsize=3
}
\lstset{literate=
  {á}{{\'a}}1 {é}{{\'e}}1 {í}{{\'i}}1 {ó}{{\'o}}1 {ú}{{\'u}}1
  {Á}{{\'A}}1 {É}{{\'E}}1 {Í}{{\'I}}1 {Ó}{{\'O}}1 {Ú}{{\'U}}1
  {à}{{\`a}}1 {è}{{\`e}}1 {ì}{{\`i}}1 {ò}{{\`o}}1 {ù}{{\`u}}1
  {À}{{\`A}}1 {È}{{\'E}}1 {Ì}{{\`I}}1 {Ò}{{\`O}}1 {Ù}{{\`U}}1
  {ä}{{\"a}}1 {ë}{{\"e}}1 {ï}{{\"i}}1 {ö}{{\"o}}1 {ü}{{\"u}}1
  {Ä}{{\"A}}1 {Ë}{{\"E}}1 {Ï}{{\"I}}1 {Ö}{{\"O}}1 {Ü}{{\"U}}1
  {â}{{\^a}}1 {ê}{{\^e}}1 {î}{{\^i}}1 {ô}{{\^o}}1 {û}{{\^u}}1
  {Â}{{\^A}}1 {Ê}{{\^E}}1 {Î}{{\^I}}1 {Ô}{{\^O}}1 {Û}{{\^U}}1
  {œ}{{\oe}}1 {Œ}{{\OE}}1 {æ}{{\ae}}1 {Æ}{{\AE}}1 {ß}{{\ss}}1
  {ű}{{\H{u}}}1 {Ű}{{\H{U}}}1 {ő}{{\H{o}}}1 {Ő}{{\H{O}}}1
  {ç}{{\c c}}1 {Ç}{{\c C}}1 {ø}{{\o}}1 {å}{{\r a}}1 {Å}{{\r A}}1
  {€}{{\EUR}}1 {£}{{\pounds}}1
}
\newcommand{\pic}[3]{
	\begin{figure}[H]
		\begin{center}
			\includegraphics[scale=#2]{doc/#1}
		\end{center}
		\caption{#3}
	\end{figure}
}
\newcommand{\piclab}[4]{
	\begin{figure}[H]
		\begin{center}
			\includegraphics[scale=#2]{doc/#1}
		\end{center}
		\caption{\label{#4}#3}
	\end{figure}
}
\newcommand{\picsidelab}[6]{
	\begin{wrapfigure}{#5}{#6}
		\includegraphics[scale=#2]{doc/#1}
		\caption{\label{#4}#3}
	\end{wrapfigure}
}
\makeindex

\title[Projet sac à dos]{Projet tutoré 2: Sac à dos}

\projet{S4}

\author{Thomas Couchoud\\
\noindent[\url{thomas.couchoud@etu.univ-tours.fr}]\\
Victor Coleau\\
\noindent[\url{victor.coleau@etu.univ-tours.fr}]}

\encadrant{Yannick Kergosien\\ %
\noindent[\url{yannick.kergosien@univ-tours.fr}]~\\
Polytech Tours\\
Département DI\\~ %
}

%%%%%%%%%%%%%%%%%%%%%%%%%%%%%%%%%%%%%%%%%%%%%%%%%%%%%%%%%%%%%%%%%%%%%%%%%%%%%%%%%%%%%%%%%%
\begin{document}

\maketitle

\pagenumbering{roman}
\setcounter{page}{0}
{
\setlength{\parskip}{0em}

\tableofcontents

%\listoffigures
%rq1 : si vous n'avez pratiquement pas de figures, laissez la ligne précédente en commentaire

%\listoftables
%rq1 : si vous n'avez pratiquement pas de tables, laissez la ligne précédente en commentaire
}


\start
%%%%%%%%%%%%%%%%%%%%%%%%%%%%%%%%%%%%%%%%%%%%%%%%%%%%%%%%%%%%%%%%%%%%%%%%%%%%%%%%%%%%%%%%%%

\chapter*{Introduction}


%--------------------------------------------------------------------------------
\chapter*{Conclusion}
\addcontentsline{toc}{chapter}{\numberline{}Conclusion}
\markboth{Conclusion}{}

\label{sec:conclusion}




%--------------------------------------------------------------------------------
%si on donne des annexes :
\appendix
\addcontentsline{toc}{part}{\numberline{}Annexes}

%--------------------------------------------------------------------------------

\chapter{Liens utiles\label{sec:liens_utiles}}
Voici une petite liste d'url intéressantes au sujet de ce projet :

\begin{itemize}
\item Site de Polytech'Tours: \url{www.polytech.univ-tours.fr}
\end{itemize}


%--------------------------------------------------------------------------------
%index : attention, le fichier dindex .ind doit avoir le même nom que le fichier .tex
%\printindex

%--------------------------------------------------------------------------------
%page du dos de couverture :

\resume{}

\motcles{}

\abstract{}

\keywords{}

\makedernierepage

\end{document}
